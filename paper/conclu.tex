\section{Conclusion}
We built Real Semantics, a dynamic floating-point imprecision detector that runs on a modified LLVM IR interpreter. Our tool tracks higher-precision representations of the numbers calculated by the interpreted program through the use of MPFR to calculate actual real numbers. Further, Real Semantics notifies the user of imprecision only when it substantially changes program behavior, which results in a high-level of usability. Lastly, our tool identifies precision loss events by line number and variable name in order to allow the user to more easily find and solve, if necessary, any floating-point precision errors.

In this paper, we have outlined our tool in detail, including other related work, our approach for the implementation, and the various test cases used to aid in our evaluation. We also outlined some limitations as explained in the need for future work as it relates to performance and evaluation optimizations. All in all, we have presented a tool that can help users detect floating point imprecision in programs written in C.

As computers continue to increase adoption across all levels of society and more users become programmers -- fully fledged citizens in interacting with their computers -- the need to support programmers using floating-point arithmetic grows only more urgent. We have three fruitful directions of attack right now: analysis, precise-by-construction, and verification. Verification provides the bedrock on which all other techniques can rely for assurance. Analysis supports both legacy code and explorations of custom written code. Ultimately, precise-by-construction techniques may supersede analyzers just as memory-safe languages have taken large market share from the unsafe languages.
