%-----------------------------------------------------------------------------
%
%               Template for sigplanconf LaTeX Class
%
% Name:         sigplanconf-template.tex
%
% Purpose:      A template for sigplanconf.cls, which is a LaTeX 2e class
%               file for SIGPLAN conference proceedings.
%
% Guide:        Refer to "Author's Guide to the ACM SIGPLAN Class,"
%               sigplanconf-guide.pdf
%
% Author:       Paul C. Anagnostopoulos
%               Windfall Software
%               978 371-2316
%               paul@windfall.com
%
% Created:      15 February 2005
%
%-----------------------------------------------------------------------------


\documentclass[preprint, authoryear]{sigplanconf}

% The following \documentclass options may be useful:

% preprint      Remove this option only once the paper is in final form.
% 10pt          To set in 10-point type instead of 9-point.
% 11pt          To set in 11-point type instead of 9-point.
% authoryear    To obtain author/year citation style instead of numeric.

\usepackage{amsmath}


\begin{document}

\special{papersize=8.5in,11in}
\setlength{\pdfpageheight}{\paperheight}
\setlength{\pdfpagewidth}{\paperwidth}

\conferenceinfo{CONF 'yy}{Month d--d, 20yy, City, ST, Country} 
\copyrightyear{20yy} 
\copyrightdata{978-1-nnnn-nnnn-n/yy/mm} 
\doi{nnnnnnn.nnnnnnn}

% Uncomment one of the following two, if you are not going for the 
% traditional copyright transfer agreement.

%\exclusivelicense                % ACM gets exclusive license to publish, 
                                  % you retain copyright

%\permissiontopublish             % ACM gets nonexclusive license to publish
                                  % (paid open-access papers, 
                                  % short abstracts)

% \titlebanner{banner above paper title}        % These are ignored unless
% \preprintfooter{short description of paper}   % 'preprint' option specified.

\title{Real Semantics}
% \subtitle{Subtitle Text, if any}

\authorinfo{Hannah Blumberg \and Yihe Huang \and Dan King \and Paola Mariselli}
           {Harvard University: School of Engineering and Applied Sciences}
           {hannahblumberg@college.harvard.edu
       \and yihehuang@g.harvard.edu
       \and daniel.zidan.king@gmail.com
       \and paolamariselli@fas.harvard.edu}

\maketitle

\begin{abstract}
The phrase ``floating-point number'' is a lie. Floating-point arithmetic is
fundamentally interval arithmetic wherein the interval lengths are a function of
the median point.

We propose a project to develop an automated tool that reasons about the
accuracy of floating-point arithmetic. It aims to discover instances where
the floating-point arithmetic result is not the most accurate floating-point
approximation to the real number implied by the algorithm (if interpreted with
real number, instead of floating-point, semantics).
\end{abstract}

% \category{CR-number}{subcategory}{third-level}

% general terms are not compulsory anymore, 
% you may leave them out

\keywords
floating-point arithmetic, bug finding

\section{Introduction}

The tool will be applied to programs written in the C language that us:
\begin{itemize}
\item floating-point numbers,
\item elementary floating-point operations, and
\item transcendental floating-point operations
\end{itemize}

There are 8.4 million results for the keyword ``float'' in the C source code on
GitHub. This certainly over counts the number of files using floating-point
arithmetic, but we're confident this is a large class of programs.

We may further divide the class of floating-point-arithmetic-using programs, a
non-exhaustive list follows.
\begin{itemize}
\item physical and biological simulations
\item video games, both for graphics and physics
\item signal processing applications, such as image and sound processing
\item embedded real-time control systems
\item embedded sensor systems
\end{itemize}

\section{The Ideal Product}

The ideal product of our research is an LLVM interpreter, modified such that:

\begin{itemize}
\item each floating-point number has a real number companion
\item floating-point operations (both elementary and transcendental) induce real
  number operations on the companions
\item a warning is issued for each floating-point output that is not the best
  approximation to its real number companion
\end{itemize}

Moreover, the interpreter should be sufficiently efficient for use in a major
software company's build cycle. Consider for example Facebook's standards as
described in the INFER paper~\cite{infer}. Maintaining this performance may
require implementing performance heuristics that leverage real number semantics,
such as replacing triangle number sums.

Finally, the project will generate a corpus of in-the-wild floating-point
arithmetic errors.

\section{The Minimal Product}

The minimal product of our research is an LLVM interpreter, modified such that:

\begin{itemize}
\item every floating-point number has a real number companion
\item elementary floating-point operations induce rational number operations on the
  companions
\end{itemize}

\section{Possibility of Failure}

We are most concerned about modifying the LLVM interpreter because we
collectively have no experience with LLVM. We are also concerned with:

\begin{itemize}
\item finding a sufficient large and interesting corpus of programs
\item finding non-trivial, correctness-relevant floating-point arithmetic bugs
\end{itemize}

\section{Technologies}

We will not implement a new interpreter nor a new C compiler. We will directly
analyze LLVM bytecode, reusing an existing LLVM interpreter. Moreover, we will
not implement a new Real number library for C. We will use RealLib, a GNU LGPL
licensed library for performing both elementary and transcendental real number
operations.

\section{Related Work}

If a single paper must be chosen, we believe the class as a group of programmers
would benefit from reading the first few sections of \emph{What Every Computer
  Scientist Should Know About Floating-Point Arithmetic} by David
Goldberg~\cite{goldberg91}.

\section{The One-Week Project}

Our one week project is the production of a modified LLVM interpreter that uses
real number semantics (for elementary operations) instead of floating-point
semantics. This exercises our ability to correctly link with RealLib. It also
provides us with an almost complete foundation of LLVM infrastructure for
continuation with the multiple week project.

\section{Collaboration}

We have divided our one-week project into two major pieces:

\begin{itemize}
\item get the RealLib library working and generate a simple test suite
\item modify the LLVM interpreter to use real number semantics
\end{itemize}

We've decided that Dan and Paola will work on the first bullet while Yihe and
Hannah concurrently work on the second bullet.

Our ideal, final project consists of a number of pieces:

\begin{itemize}
\item modifications to the LLVM interpreter to call RealLib
\item generating a set test of test cases to debug our interpreter
\item gathering a corpus of floating-point-arithmetic-using programs
\item running experiments on the full corpus
\item writing the report, generating figures, computing statistics, and
  summarizing background knowledge
\end{itemize}

\acks

We acknowledge Eddie's face

% We recommend abbrvnat bibliography style.

\bibliographystyle{abbrvnat}

\bibliography{proposal}

% The bibliography should be embedded for final submission.

\end{document}

%                       Revision History
%                       -------- -------
%  Date         Person  Ver.    Change
%  ----         ------  ----    ------

%  2013.06.29   TU      0.1--4  comments on permission/copyright notices

